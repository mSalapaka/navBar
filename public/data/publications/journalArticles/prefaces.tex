\documentstyle[11pt,epsf]{article}
\setlength{\oddsidemargin}{0in}
\setlength{\evensidemargin}{0in}
\setlength{\topmargin}{0in}

 \addtolength{\textwidth}{1.2in}
\addtolength{\textheight}{.8in}


\title{}
\newtheorem{theorem}{Theorem}
\newtheorem{corollary}{Corollary}
\newtheorem{lemma}{Lemma}
\newtheorem{claim}{Claim}
\newtheorem{definition}{Definition}
\newtheorem{assum}{Assumption}
\newcommand{\normone}[1]{\mbox{$\parallel #1 \parallel _{1}$}}
\newcommand{\normtwo}[1]{\mbox{$\parallel #1 \parallel _{2}$}}
\newcommand{\norminf}[1]{\mbox{$\parallel #1 \parallel _{\infty}$}}
\newcommand{\hinf}{{\cal{H}}_\infty}
\newcommand{\hdio}{{\cal{H}}_2}
\newcommand{\qed}{\hfill\vrule height 1.6ex width 1.5ex depth -.1ex}
\newcommand{\soh}{\mbox{$\ell_1^{n_z\times n_w}$}}
\newcommand{\sohp}{\mbox{$\ell_2^{n_z\times n_w}$}}
\newcommand{\soq}{\mbox{$\ell_1^{n_u\times n_y}$}}
\newcommand{\dims}[2]{\mbox{$\ell_1^{#1\times#2}$}}
\newcommand{\nfrac}[2]{\mbox{$\frac{\textstyle #1}{\textstyle #2}$}}
\newcommand{\te}{~  \mbox{there exists}~}
\newcommand{\st}{~ \mbox{such that} ~}
\newcommand{\nd}{~  \mbox{and} ~ }
\newcommand{\fa}{~  \mbox{for all} ~ }
\newcommand{\as}{~  \mbox{as} ~ }
\newcommand{\with}{~  \mbox{with} ~ }
\newcommand{\parder}[2]{\mbox{$\frac{\partial  #1}{\partial  #2}$}}
\newcommand{\Parder}[2]{\mbox{$\frac{D #1}{D  #2}$}}
\newcommand{\wbar}{\mbox{$\overline{\omega}$}}
\newcommand{\R}{R}
\date{}
\begin{document}
\begin{center}
{\bf Yadav, V.,   M.V. Salapaka, P. G. Voulgaris, Architectures for Distributed Controller with Sub-controller Communication Uncertainty, IEEE Trans. Automatic Control, �accepted, 2009}
\end{center}

In this paper, the major insight was that by simply imposing structure on Multiple-Input, Multiple-Output {\it Transfer Function} matrices that incorporate communication constraints in a controller design even stability is not guaranteed; there is a non-unique way of implementing the structure and certain structures can lead to instability. Methods were devised to search in an optimization framework over sub-classes of structures that lead to stable systems.

The first author was a PhD student of M. V. Salapaka.  The contribution of M. V. Salapaka was a lead contribution.


\newpage 
\begin{center}
{\bf P. Agarwal, M. V. Salapaka, Real time estimation of equivalent cantilever parameters in tapping mode atomic force microscopy, Applied Physics Letters Vol. 95, 083113, 2009}
\end{center}

Prior to this paper, tapping-mode AFM was primarily employed for topography estimation. With the experimental as well as theoretical contributions of this work, it is now possible to estimate equivalent cantilever parameters; variables that are often utilized to obtain properties (like elasticity) of the sample at the nanoscale, while obtaining the topography image using tapping mode. A patent disclosure is approved and the product is being explored for being licensed commercially. The paper was selected to the {\it Virtual Journal of Nanotechnology}.

The first author was a PhD student of M. V. Salapaka.  The contribution of M. V. Salapaka was a major contribution.




\newpage 
\begin{center}
{\bf P. Agarwal, T. De, M. V. Salapaka, Real time reduction of probe-loss using switching gain controller for high speed atomic force microscopy, Review of Scientific Instruments,  vol. 80,pp. 103701, 2009}
\end{center}

Prior to this paper, controller gains were not selectively chosen based on whether the probe in a atomic force microscopy setting was interacting with the sample or not thereby greatly reducing the bandwidth of operation. Using the probe-loss signal that was invented in M. V. Salapaka's  lab, controllers gains were chosen to have the dual advantage of gentle, high resolution on-sample behavior while simultaneously reducing the probe-loss affected areas.  The paper was selected to the {\it Virtual Journal of Nanotechnology}.


The first two authors are PhD students of M. V. Salapaka.  The contribution of M. V. Salapaka was a major contribution.

\newpage

\begin{center}
{\bf Hullas Sehgal, Tanuj Aggarwal, and Murti V. Salapaka, High bandwidth force estimation for optical tweezers, Appl. Phys. Lett. 94, 153114, 2009}
\end{center}

In this paper we demonstrated that using systems techniques one can significantly enhance the bandwidth of optical tweezer operation. This is the also first paper on optical tweezer from M. V. Salapaka's lab and is  a demonstration of the optical tweezer experimental capabilities of his lab. 


The first two authors are PhD students of M. V. Salapaka.  The contribution of M. V. Salapaka was a major contribution.
\newpage 
\begin{center}{\bf S. Salapaka and M. V. Salapaka, Scanning Probe Microscopy, A Controls Systems Perspective on Nanointerrogation,Vol. 28, No. 2, pp. 65-83, 2008}\end{center}

An overview of the systems and controls perspectives to scanning probe microscopy was presented. The contribution in the actual conceptual writing of the content of the paper was equal but the lead author initiated the paper which  involved contacting the Editor, organizing the structure of the paper and being the primary responsible for many iterates of the paper (which was considerable work). 
\newpage
\begin{center}{\bf M. Basso, D. Materassi, M. V. Salapaka, Hysteresis Models of Dynamic Mode Atomic Force Microscopes: Analysis and Identification", Nonlinear Dynamics, Vol. 54, No 4, pp 297-306, 2008 }\end{center}
A new means of identifying dynamics was devised. The major conceptual contribution were of the first two authors. M. V. Salapaka played a guiding role and provided strong experimental basis/insights for the paper
\newpage
\begin{center}{\bf A. Sebastian, A. Gannepalli, A., M.V Salapaka., "A Review of the Systems Approach to the Analysis of Dynamic-Mode Atomic Force Microscopy," IEEE Transactions on Control Systems Technology, , vol.15, no.5, pp.952-959, Sept. 2007 }
\end{center}

The title has the word review. However, in this paper we also present methods on using integral quadratic constraints to estimate higher harmonics; thus bringing a very recent control theoretical tool to an application in Atomic Force Microscope.

The first two authors are PhD students of M. V. Salapaka.  The contribution of M. V. Salapaka was a major contribution.
\newpage
\begin{center}{\bf Tathagata De, Pranav Agarwal, Deepak R. Sahoo, and Murti V. Salapaka, "Real-time detection of probe loss in atomic force microscopy " Appl. Phys. Lett. 89, 133119 (2006)}
\end{center}
With this method it is now possible to evaluate in real-time whether data in an Atomic Force Microscopy image is real or spurious.

The first three authors are PhD students of M. V. Salapaka.  The contribution of M. V. Salapaka was a major contribution.
\newpage
\begin{center}{\bf A.G. Hatch, R.C. Smith, T. De, and M.V. Salapaka. "Construction and experimental implementation of a model-based inverse filter to attenuate hysteresis in ferroelectric transducers", IEEE Transactions on Control Systems Technology. Volume 14,  Issue 6,  Nov. 2006 Page(s):1058 - 1069 }
\end{center}
This project was under a jointly funded NSF grant  (with the other PI being Ralph Smith) where the inverse models of hysteresis were to be employed for Atomic force microscopy work. 

The theoretical constructs were primarily achieved by  from Prof. Smith with the experimental part and subsequent suggestions for changing modeling strategy achieved primarily in M. V. Salapaka's lab. T. De is M.V.Salapaka's student and A. G. Hatch was R. Smith's student. 
\newpage
\begin{center}{\bf R.C. Smith, A.G. Hatch, T. De, M.V. Salapaka, R.C.H. del Rosario, and J.K. Raye. "Model development for atomic force microscope stage mechanisms". SIAM Journal on Applied Mathematics, vol. 66 no. 6, 2006. }
\end{center}
This project was under a jointly funded NSF grant  (with the other PI being Ralph Smith) where the inverse models of hysteresis were to be employed for Atomic force microscopy work.

 The theoretical constructs were primarily achieved by  by  Prof. Smith with collaboration with R.C.H. del Rosario, and J.K. Raye with  the experimental part and subsequent suggestions for changing modeling strategy achieved primarily in M. V. Salapaka's lab. T. De is M.V.Salapaka's student and A. G. Hatch was R. Smith's student. 
\newpage
\begin{center}{\bf 	D. R. Sahoo(s) and A. Sebastian and M. V. Salapaka, "Harnessing the transient signals in atomic force microscopy" International Journal of Robust and Nonlinear Control, Special Issue on Nanotechnology and Micro-biology,  Vol. 15, pp: 805-820, 2005.}\end{center}
This paper developed on an earlier paper and showed the real-time efficacy of the transient force atomic force microscopy scheme. Concepts from communications area on channel equalization proved quite crucial.

The first two authors are PhD students of M. V. Salapaka. 

\newpage
\begin{center}{\bf Anil Gannepalli and Abu Sebastian and Jason Cleveland and Murti V. Salapaka, �Thermally driven non-contact atomic force microscopy�, Applied Physics Letters, 87,(11), 111901, 2005 }
\end{center}
In this paper, the {\it least} invasive method of frequency modulated Atomic Force Microscopy was reported. It was highlighted in Nature.


The first two authors are PhD students of M. V. Salapaka. Jason Cleveland's role was more advisory.  The contribution of M. V. Salapaka was a major contribution.

\newpage
\begin{center}{\bf Xin Qi, M. V. Salapaka, Petros G. Voulgaris and Mustafa Khammash, �Structured optimal and robust control with multiple criteria: A convex solution�, IEEE Trans. Automatic Control,49 (10) pp 1623-1640, October 2004. }
\end{center}
In this paper, a convex solution to structured control design with performance specifications under diverse metrics was achieved. 

Xin, Qi was a joint PhD student shared with Mustafa Khammash. The last three authors contributed equally. 
\newpage
\begin{center}{\bf D. R. Sahoo(s), A. Sebastian,  M.V. Salapaka, �Transient Signal based sample-detection in Atomic Force Microscopy�, Applied Physics Letters, 83(26), pp. 5521-5523, December (2003) }
\end{center}
This paper presented a entirely new means of imaging that surpassed perceived limitation of Atomic Force Microscopy based imaging and demonstrated that imaging can be achieved while the probe is in transient. Concepts from communications area on channel equalization proved quite crucial.

The first two authors are PhD students of M. V. Salapaka. 
\newpage
\begin{center}{\bf S. Salapaka, A. Sebastian, J. P. Cleveland and M. V. Salapaka, "High Bandwidth Nano-positioner: A Robust Control Approach", Review of Scientific Instruments, Vol. 73, no. 9, pp. 3232-3241.  }
\end{center}
This was the first reportage of modern control tools applied to later nanopositioning.  The work was done at M. V. Salapaka's lab.

 The first author (now a faculty at UIUC)  was a student during this work. J.P. Cleveland (founder of Asylum Research) played an advisory role.
\newpage
\begin{center}{\bf M. Sridharan , M. V. Salapaka, A. Somani,, �A Practical Approach to Operating Survivable WDM Networks�, Special Issue on WDM Based Network Architectures, IEEE Journal on Selected Areas in Communications on WDM-based Network Architectures. vol. 20, no. 1, January 2002.}
\end{center}
The application domain  expertise was primarily provided by A. Somani. M. V. Salapaka was the main responsible for guiding the student M. Sridharan  (he was A. Somani's student) on the optimization process which was the central focus of the paper. 
\newpage
\begin{center}{\bf M. Sridharan, A. Somani, M. V. Salapaka , �Approaches for Capacity and Revenue Optimization in Survivable WDM Networks�,  Spl. Issue on Survivable Optical Networks, Journal of High Speed Networks, August 2001. vol. 10, no. 2, pp. 109 -125. }
\end{center}
The application domain  expertise was primarily provided by A. Somani. M. V. Salapaka was the main responsible for guiding the student M. Sridharan  (he was A. Somani's student) on the optimization process which was the central focus of the paper. 
\newpage
\begin{center}{\bf A. Sebastian, M. V. Salapaka, D. J. Chen, J. P. Cleveland, "Harmonic and power balance tools for tapping �mode AFM", Journal of Applied Physics, June 2001, Vol. 89, Nov. 11, pages. 6473-6480. }\end{center}

A new concept of separating the cantilever system as a Linear Time Invariant system interacting with a static memoryless nonlinearity was developed. This has had significant impact on identification techniques and a way of simplifying the complexity of imaging in atomic force microscopes. 


The first author was a  PhD student of M. V. Salapaka. Jason Cleveland's role was advisory.  The contribution of M. V. Salapaka was a major contribution amongst the senior authors.
\newpage
\begin{center}{\bf M. Khammash, M.V. Salapaka, T. VanVoorhis, "Robust Synthesis in l1: A globally optimal solution",  IEEE Trans. Automatic Control, Volume: 46 Issue: 11 , Nov. 2001, Page(s): 1744 -1754.}\end{center}
In this paper, a globally optimal solution (with the proof that the optimal within a pre-specified tolerance is achieved) was devised for a hard non-convex problem (the so-called robust performance problem. 

The authors names are arranged alphabetically and the first two authors were the prime responsible for writing the paper.  

\newpage
\begin{center}{\bf M.V. Salapaka, D. Chen, J. Cleveland, "Linearity of amplitude and phase in tapping-mode atomic force microscopy", Physical Review B., Vol 61, No. 2, Pages: 1106-1115, 2000 }\end{center}
In this paper using dynamical systems tools, the experimentally observed features of linearity of amplitude and phase in Atomic Force Microscopy were developed.

The author names are arranged according to their contribution.
\newpage
\begin{center}{\bf M. V. Salapaka, M. Khammash, M. Dahleh,  �Solution of MIMO H2/l1 problem without zero interpolation�, SIAM Journal on Optimization and Control, V37, no. 6, pp.1865-1873, 1999.} \end{center}
A new means of solving an infinite-dimensional mixed objective control problem was developed. The author names are in the order of contribution. 
\newpage
\begin{center}{\bf M. Ashhab, M. V. Salapaka, M. Dahleh, and I. Mezic, "Melnikov-based dynamical analysis of microcantilevers in scanning probe microscopy", Nonlinear Dynamics, Kluwer Academic Publishers, Vol 20, pages 197-220, 1999.}\end{center}

Using Melnikov theory it was established that Atomic Force Microscopy dynamics can exhibit chaotic behavior. 

The first author is a student and the last three senior authors are arranged according to contributions. 

\newpage
\begin{center}{\bf M. Ashhab, M. V. Salapaka, M. Dahleh, and I. Mezic, "Dynamical analysis and control of micro-cantilevers", Automatica,Vol. 35, no. 10, 1663-1670 October 1999 }\end{center}

Using Melnikov theory it was established that Atomic Force Microscopy dynamics can exhibit chaotic behavior; means of controlling the dynamics to achieve desired behavior were developed.

 The first author is a student and the last three senior authors are arranged according to contributions.  

\newpage
\begin{center}{\bf M. V. Salapaka, M. Dahleh, and P. Voulgaris,  "Mimo optimal control design: the interplay of the H2 and the l1 norms", IEEE Transactions on Automatic Control, V43, no. 10:pp.1374-1388, 1998.}\end{center}
Paper addressed an infinite dimensional optimization problem that was solved with converging upper and lower bounds. 

M. V. Salapaka was a PhD student at the time of the work and was the main responsible for the entire work.
\newpage
\begin{center}{\bf M. V. Salapaka, H. S. Bergh, J. Lai, A. Majumdar, and E. McFarland, "Multimode noise analysis of cantilevers for scanning probe microscopy", Journal of Applied Physics, 81, no. 6:pp. 2480-2487, 1997}\end{center}
Paper developed the analytical power spectra due to thermal noise in Atomic Force Microscopy cantilevers. 

M. V. Salapaka was a PhD student at the time of the work and was the main responsible for the entire work.

\newpage

\begin{center}{\bf M. V. Salapaka, M. Dahleh, and P. Voulgaris, "Mixed objective control synthesis: Optimal l1/H2 control", SIAM Journal on Control and Optimization, V35 no. 5:pp. 1672--1689, 1997.}
\end{center}
In this paper it was established that the seemingly infinite dimensional optimization problem can be provably shown to be finite dimensional. Moreover an estimate on the size of the optimization problem was also provided.


M. V. Salapaka was a PhD student at the time of the work and was the main responsible for the entire work.
\newpage
\begin{center}{\bf M. V. Salapaka, P. Voulgaris, and M. Dahleh. "SISO controller design to minimize a positive combination of the l1 and the H2 norm", Automatica, V33, no. 3:pp. 387--391, March 1997.}
\end{center}
In this paper it was established that the seemingly infinite dimensional optimization problem can be provably shown to be finite dimensional. It was shown that  estimate on the size of the optimization problem is not obtainable


M. V. Salapaka was a PhD student at the time of the work and was the main responsible for the entire work.\newpage
\begin{center}{\bf M. V. Salapaka, P. Voulgaris, and M. Dahleh, "Controller design to optimize a composite performance measure", Journal of Optimization Theory and Applications, V91, no. 1:pp. 91-113, 1996.}
\end{center}
In this paper it was established that the seemingly infinite dimensional optimization problem can be provably shown to be finite dimensional. It was shown that  estimate on the size of the optimization problem is not obtainable.

M. V. Salapaka was a PhD student at the time of the work and was the main responsible for the entire work.
\newpage
\begin{center}{\bf V. Ramakrishna, M. V. Salapaka, M. Dahleh, H. Rabitz, and A. Pierce, ``Controllability of molecular systems", Physical Review A., V51, no. 2:pp. 960-966, Feb 1995. }
\end{center}


This was one of the pioneering papers that applied Lie algebra based theory developed for bilinear systems to Quantum Systems.


 M. V. Salapaka was a student at the time of the work and the main responsible for the mathematical aspects of the problem.  The first two authors were advisees of the last two authors respectively and the names appear alphabetically and had equal contribution. 
\newpage

\begin{center}
{\bf Springer Verlag Book on Multiple Objective Control Synthesis}
\end{center}



This book develops the necessary mathematical background and basic results in control theory that are needed to set up and solve a class of multiple objective control design problems. Exact solutions for certain problems are provided, as well as approximations to more general problems. This book is intended for engineering and mathematics graduate students who are interested in control theory and the application of functional analysis and optimization tools to engineering problems. 


This book is directly based on M. V. Salapaka's thesis and related work.
\end{document}
